%%%%%%%%%%%%%%%%%%%%%%%%%%%%%%%%%%%%%%%%%%%%%%%%%%%%%%%%%%%%%%%%%%%%%%%
%% Optionen zum Layout des Artikels                                  %%
%%%%%%%%%%%%%%%%%%%%%%%%%%%%%%%%%%%%%%%%%%%%%%%%%%%%%%%%%%%%%%%%%%%%%%%
\documentclass[%
a4paper,							% alle weiteren Papierformat einstellbar
oneside,							% einseitiger Druck
%twoside,							% Doppelseiten
%twocolumn,						% zweispaltiger Satz
%landscape,						% Querformat
12pt,								% Schriftgröße (12pt, 11pt (Standard))
%BCOR1cm,							% Bindekorrektur, bspw. 1 cm
%DIVcalc,							% führt die Satzspiegelberechnung neu aus
%halfparskip*,				% Absatzformatierung s. scrguide 3.1
headsepline,					% Trennline zum Seitenkopf	
footsepline,					% Trennline zum Seitenfuß
titlepage,						% Titelei auf eigener Seite
%normalheadings,			% Überschriften etwas kleiner (smallheadings)
%leqno,   						% Nummerierung von Gleichungen links
%fleqn,								% Ausgabe von Gleichungen linksbündig
%draft								% überlangen Zeilen in Ausgabe gekennzeichnet
toc=listof,						% Abb.-Verzeichnis im Inhalt
toc=bibliography			% Tab.-Verzeichnis im Innhalt
]
{scrbook}


%% Deutsche Anpassungen %%%%%%%%%%%%%%%%%%%%%%%%%%%%%%%%%%%%%
\usepackage[ngerman]{babel}
\usepackage[ngerman]{translator}
\usepackage[T1]{fontenc}
\usepackage[utf8]{inputenc} %UTF-8 als Charset
\usepackage{lmodern} % Type1-Schriftart für nicht-englische Texte
\usepackage[autostyle=true,german=quotes]{csquotes}





%% Globale Variablen und Funktionen %%%%%%%%%%%%%%%%%%%%%%%%%%%%%%%%%%%%%
\renewcommand{\author}{Autor}
\renewcommand{\title}{Ein geeigneter Titel}
\newcommand{\uni}{Duale Hochschule Baden-Württemberg Mannheim}
\newcommand{\art}{Bachelorarbeit}
\newcommand{\studiengang}{International Management for Business and Information Technology}
\newcommand{\ort}{Mannheim}
\newcommand{\abgabedatum}{xx.xx.xx}
\newcommand{\kurs}{Kurs}
\newcommand{\matrikelnummer}{M-Nr.}
\newcommand{\firma}{Unternehmen}
\newcommand{\firmenort}{Firmenort}
\newcommand{\firmenbetreuer}{Betreuer}
\newcommand{\hochschulbetreuer}{Betreuer}


% freie Seiten ohne Seitencounter zu erhöhen
\newcommand\blankpage{%
	\newpage
	\thispagestyle{empty}
	\mbox{}
	\addtocounter{page}{-1}
	\newpage
}

% Abstractumgebung definieren
\newenvironment{abstract}%
    {\thispagestyle{empty}\null\vfill\begin{center}%
    \bfseries\abstractname\end{center}}%
    {\vfill\null}

%% allgemeine Paketimports und Einstellungen %%%%%%%%%%%%%%%%%%%%%%%%%%%%%%%%%%%%%


\usepackage{color}
\definecolor{LinkColor}{rgb}{0,0,0}
\definecolor{ListingBackground}{rgb}{0.92,0.92,0.92}

% Verlinkunggen innheralb der PDF, PDF Attribute, etc. 
\usepackage[%
	pdftitle={\title},
	pdfauthor={\author},
	pdfsubject={\art},
	pdfcreator={pdflatex, LaTeX with KOMA-Script},
	pdfpagemode=UseOutlines, % Beim Oeffnen Inhaltsverzeichnis anzeigen
	pdfdisplaydoctitle=true, % Dokumenttitel statt Dateiname anzeigen.
	pdflang=de % Sprache des Dokuments.
]{hyperref} 

% (Farb-)einstellungen für die Links im PDF
\hypersetup{%
	colorlinks=true, % Aktivieren von farbigen Links im Dokument
	linkcolor=LinkColor, % Farbe festlegen
	citecolor=LinkColor,
	filecolor=LinkColor,
	menucolor=LinkColor,
	urlcolor=LinkColor,
	bookmarksnumbered=false % Überschriftsnummerierung im PDF Inhalt anzeigen.
}

\usepackage{bookmark}
\bookmarksetup{
  numbered, 
  open,
}

% Einstellungen für das Inhaltsverzeichnis
\usepackage{tocstyle}
\newtocstyle[KOMAlike][leaders]{alldotted}{}
\usetocstyle{alldotted}

% Glossar und Abkürzungsverzeichnis, Eintrag in TOC, keine Seitenzahlen hinter Eintrag
\usepackage[toc, acronym, nonumberlist]{glossaries}
\makeindex
\makeglossaries
\loadglsentries {content/glossar}
\loadglsentries {content/acronyme}



\usepackage[onehalfspacing]{setspace} %Zeilenabstand setzen

\usepackage{geometry} %Seitenränder
\geometry{a4paper, left=30mm, right=25mm,}

\usepackage{graphicx} % Zum Laden von Grafiken

\usepackage[%
	style=authoryear,	%Harvard Style
	backend=biber,			%Biber backend für UTF8 Support
	isbn=false,
	doi=false
]{biblatex} % Bibliographiestil und Package
\bibliography{content/bibliographie}	% Literaturverzeichnis


\clubpenalty = 10000 
\widowpenalty = 10000 
\displaywidowpenalty = 10000

\begin{document}

\pagestyle{empty} %%Keine Kopf-/Fusszeilen auf den ersten Seiten.


%%%%%%%%%%%%%%%%%%%%%%%%%%%%%%%%%%%%%%%%%%%%%%%%%%%%%%%%%%%%%%%%%%%%%%%
%% Ihr Artikel                                                       %%
%%%%%%%%%%%%%%%%%%%%%%%%%%%%%%%%%%%%%%%%%%%%%%%%%%%%%%%%%%%%%%%%%%%%%%%

\pdfbookmark[section]{Titelseite}{title}
\begin{titlepage}

\newgeometry{
	top=3cm,
	bottom=2cm
}

\newcommand{\HRule}{\rule{\linewidth}{0.5mm}} % Defines a new command for the horizontal lines, change thickness here

 
%----------------------------------------------------------------------------------------
%	HEADING SECTIONS
%----------------------------------------------------------------------------------------

\begin{minipage}[c]{0.5\textwidth}
\includegraphics[height=3.2cm]{images/dhbw.png}
\end{minipage}
\hspace{.4cm}
\begin{minipage}[c]{0.5\textwidth}
\includegraphics[height=3.2cm]{images/firma.png}
\end{minipage}

\center % Center everything on the page

\vspace{1cm}
\textsc{\LARGE \uni}\\[.5cm] % Name of your university/college
%----------------------------------------------------------------------------------------
%	TITLE SECTION
%----------------------------------------------------------------------------------------

\HRule \\[.4cm]
{ \huge \bfseries \title}\\ % Title of your document
\HRule \\[.4cm]
\vspace{.5cm}
\textsc{\Large \art}\\[.5cm]
\textit{im Studiengang}\\
\textsc{{\Large \studiengang}}\\[0.5cm] % Major heading such as course name
\textit{von}\\
\textsc{\Large \author}\\[2cm]
\normalsize
%----------------------------------------------------------------------------------------
%	AUTHOR SECTION
%----------------------------------------------------------------------------------------
\begin{tabbing}
mmmmmmmmmmmmmmmmmmmmmmmmmm     \= \kill
		\textbf{Matrikelnummer, Kurs}			 \> \matrikelnummer, \kurs\\
		\textbf{Firma}      \> \firma, \firmenort\\
		\textbf{Firmenbetreuer}              \> \firmenbetreuer\\
		\textbf{Hochschulbetreuer}             \>  \hochschulbetreuer
\end{tabbing}
%----------------------------------------------------------------------------------------
%	DATE SECTION
%----------------------------------------------------------------------------------------
\vspace{1cm}
{\large \abgabedatum}\\ % Date, change the \today to a set date if you want to be precise


\vfill % Fill the rest of the page with whitespace

\end{titlepage}
\restoregeometry

% Pagenumbering auf römisch für die Preambel
\renewcommand{\thepage}{\Roman{page}}
\setcounter{page}{1}

%% Sperrvermerk %%%%%%%%%%%%%%%%%%%%%%%%%%%%%%%%%%%%%%%
\phantomsection
\addcontentsline{toc}{chapter}{Sperrvermerk}
\section*{Sperrvermerk}
\vspace*{2em}
Die vorliegende \art
\begin{center}
"`\title"'
\end{center}
enthält unternehmensinterne bzw. vertrauliche Informationen der \firma mit Sitz in Weinheim. Es ist untersagt:
\begin{itemize}
	\item den Inhalt dieser Arbeit (einschließlich Daten, Abbildungen, Tabellen,
Zeichnungen usw.) als Ganzes oder auszugsweise weiterzugeben,
	\item Kopien oder Abschriften dieser Arbeit (einschließlich Daten, Abbildungen,
Tabellen, Zeichnungen usw.) als Ganzes oder in Auszügen anzufertigen,
\item diese Arbeit zu veröffentlichen bzw. digital, elektronisch oder virtuell zur
Verfügung zu stellen.
\end{itemize}
Ausnahmen bedürfen der schriftlichen Genehmigung durch den Verfasser und der \firma.

\vspace{3em}

\ort, \abgabedatum
\vspace{4em}

\author
\newpage

%% Ehrenwörtliche Erklärung %%%%%%%%%%%%%%%%%%%%%%%%%%%%%%%%%%%%%%%
\phantomsection
\addcontentsline{toc}{chapter}{Ehrenwörtliche Erklärung}
\section*{Ehrenwörtliche Erklärung}
\vspace*{2em}

Hiermit erkläre ich ehrenwörtlich, \\
\begin{itemize}
\item  dass ich die vorliegende Bachelorarbeit zum Thema "`\title"' ohne fremde Hilfe angefertigt habe,
\item dass ich die aus fremden Quellen direkt oder indirekt übernommenen Gedanken an den entsprechenden Stellen innerhalb der Arbeit als solche kenntlich gemacht habe,
\item dass ich meine Projektarbeit bisher keiner anderen Prüfungsbehörde vorgelegt und auch noch nicht veröffentlicht habe,
\item dass ich meine Projektarbeit bisher keiner anderen Prüfungsbehörde vorgelegt und auch noch nicht veröffentlicht habe.
\end{itemize}

Ich bin mir bewusst, dass eine unwahre Erklärung rechtliche Folgen haben wird.

\vspace{3em}

\ort, \abgabedatum
\vspace{4em}

\author
\newpage


%% Abstract %%%%%%%%%%%%%%%%%%%%%%%%%%%%%%%%%%%%%%%
\phantomsection
\addcontentsline{toc}{chapter}{Abstract}
\renewcommand\abstractname{Abstract} % Überschrift des Abstracts
\begin{abstract}
	\input{content/abstract}
\end{abstract}
\newpage


%% Inhaltsverzeichnis %%%%%%%%%%%%%%%%%%%%%%%%%%%%%%%%%%%%%%%
\phantomsection
\addcontentsline{toc}{chapter}{Inhaltsverzeichnis}
\tableofcontents			% Inhaltsverzeichnis


% Pagenumbering auf arabisch für den Hauptteil
\newpage
\renewcommand{\thepage}{\arabic{page}}
\setcounter{page}{1}

%\pagestyle{empty}		% keine Kopf und Fußzeile (k. Seitenzahl)
\pagestyle{headings}	% lebender Kolumnentitel  


%% eigentlicher Inhalt %%%%%%%%%%%%%%%%%%%%%%%%%%%%%%%%%%%%%%%
\chapter{Einführung}
\section{Motivation}
Ein \gls{GE} wird so erstellt. Eine \gls{AK} ähnlich. Außerdem gibt es noch \glspl{QVG}. Hier eine Zitat \parencite[vgl.][S. 25]{Beispiel.2015}.
\section{Aufbau und Zielsetzung der Arbeit}

\section{Abgrenzung}




%% Appendix %%%%%%%%%%%%%%%%%%%%%%%%%%%%%%%%%%%%%%%

% Pagenumbering wieder auf römisch für Appendix
\newpage
\renewcommand{\thepage}{\Roman{page}}
\setcounter{page}{4}

%% Verzeichnisse %%%%%%%%%%%%%%%%%%%%%%%%%%
\listoftables				% Tabellenverzeichnis
\listoffigures				% Abbildungsverzeichnis

% getrennte Quellenverzeichnisse
% für ein einzelnes Quellenverzeichnis nur \printbibliography ohne Optionen verwenden
% Filter für die "`Hauptbibliografie"' bauen
% !Die in diesem Filter genannten types unten als 'nottype' aufführen
\defbibfilter{main}{
	type=article
	or type=book
}

\printbibliography[filter=main] % Hauptbibliografie drucken
\printbibliography[title={Sonstige Quellen}, nottype=book, nottype=article] % Sonstige Quellen drucken


\printglossary[type=\acronymtype,title=Abkürzungsverzeichnis]
\printglossary[type=main,title=Glossar]

\end{document}